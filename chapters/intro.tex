\documentclass[output=paper]{../langscibook}
\author{Amanda Edmonds\affiliation{Université Paul-Valéry Montpellier 3}\orcid{}\and Pascale Leclercq\affiliation{Université Paul-Valéry Montpellier 3}\orcid{}\lastand Aarnes Gudmestad\affiliation{Virginia Polytechnic Institute and State University}\orcid{}}
\title{Introduction: Reflecting on data interpretation in SLA }
\abstract{}
\IfFileExists{../localcommands.tex}{
  \addbibresource{../localbibliography.bib}
  % add all extra packages you need to load to this file

\usepackage{tabularx,multicol}
\usepackage{url}
\urlstyle{same}

\usepackage{enumitem}

\usepackage{listings}
\lstset{basicstyle=\ttfamily,tabsize=2,breaklines=true}

\usepackage{langsci-basic}
\usepackage{./langsci-optional}
\usepackage{langsci-lgr}
\usepackage{langsci-gb4e}

\usepackage{jambox}

\newfontfamily\cjkfont
  [Scale=MatchLowercase,BoldFont=SourceHanSerifSC-Bold.otf]{SourceHanSerifSC-Regular.otf}
\AdditionalFontImprint{Source Han Serif}


\usepackage{siunitx}
\sisetup{output-decimal-marker={.},detect-weight=true, detect-family=true, detect-all, input-symbols={\%}, free-standing-units, input-open-uncertainty= , input-close-uncertainty= ,table-align-text-pre=false,uncertainty-separator={\,},group-digits=false,detect-inline-weight=math}
\DeclareSIUnit[number-unit-product={}]{\percent}{\%}
\makeatletter \def\new@fontshape{} \makeatother
\robustify\bfseries % For detect weight to work

  \newcommand*{\orcid}{}

\renewcommand{\sectref}[1]{Section~\ref{#1}}

% \renewcommand{\lsCoverTitleFont}[1]{\sffamily\addfontfeatures{Scale=MatchUppercase}\fontsize{41pt}{15mm}\selectfont #1}
% \renewcommand{\lsChapterFooterSize}{\scriptsize}

\makeatletter
\let\thetitle\@title
\let\theauthor\@author
\makeatother

\newcommand{\togglepaper}[1][0]{
%   \bibliography{../localbibliography}
  \papernote{\scriptsize\normalfont
    \theauthor.
    \thetitle.
    To appear in:
    Change Volume Editor \& in localcommands.tex
    Change volume title in localcommands.tex
    Berlin: Language Science Press. [preliminary page numbering]
  }
  \pagenumbering{roman}
  \setcounter{chapter}{#1}
  \addtocounter{chapter}{-1}
}

\newcommand{\keywords}[1]{\noindent\bfseries Keywords: \MakeCapital#1}

\let\oldtabularx\tabularx	% number in tabulars
    \let\endoldtabularx\endtabularx
    \renewenvironment{tabularx}{\normalfont\addfontfeatures{Numbers={Monospaced,Lining}}\selectfont\oldtabularx}{\endoldtabularx}


\newcommand{\NS}{\hphantom{N}{NS}:}
\newcommand{\TRS}{\hphantom{NNS:}~}
\newcommand{\NNS}{NNS:}
 
  %% hyphenation points for line breaks
%% Normally, automatic hyphenation in LaTeX is very good
%% If a word is mis-hyphenated, add it to this file
%%
%% add information to TeX file before \begin{document} with:
%% %% hyphenation points for line breaks
%% Normally, automatic hyphenation in LaTeX is very good
%% If a word is mis-hyphenated, add it to this file
%%
%% add information to TeX file before \begin{document} with:
%% %% hyphenation points for line breaks
%% Normally, automatic hyphenation in LaTeX is very good
%% If a word is mis-hyphenated, add it to this file
%%
%% add information to TeX file before \begin{document} with:
%% \include{localhyphenation}
\hyphenation{
affri-ca-te
affri-ca-tes 
Berg-green
Jap-a-nese
Gram-mat-i-cal-i-ty
Mac-Whin-ney
Lec-lercq
meth-od-o-log-i-cal
}

\hyphenation{
affri-ca-te
affri-ca-tes 
Berg-green
Jap-a-nese
Gram-mat-i-cal-i-ty
Mac-Whin-ney
Lec-lercq
meth-od-o-log-i-cal
}

\hyphenation{
affri-ca-te
affri-ca-tes 
Berg-green
Jap-a-nese
Gram-mat-i-cal-i-ty
Mac-Whin-ney
Lec-lercq
meth-od-o-log-i-cal
}
 
  \togglepaper[1]%%chapternumber
}{}

\begin{document}
\maketitle 
%\shorttitlerunninghead{}%%use this for an abridged title in the page headers


\noindent The past decade has seen a growing number of publications that urge researchers in the field of second language acquisition (SLA) to engage more directly and more critically with questions of research methodology. These include, among many others, \citet{Plonsky2014}, who makes clear recommendations for quantitative second-language (L2) research and issues a call for change, \citet{LeclercqEtAl2014}, who call for more transparency in the assessment of L2 proficiency, \citet{MarsdenEtAl2016}, who make a strong case for the importance of replication in moving the field forward, \citet{GudmestadEdmonds2018}, who showcase different ways to bring critical reflections on method to the fore, and \citet{Ortega2014ways}, who draws attention to the need to move beyond a native-speaker bias in L2 research. Although diverse in aim and scope, these endeavours and others like them share a strong interest in moving methodological practices forward. \citet[825]{Byrnes2013} goes so far as to characterise this increasing interest as a “methodological turn” within our field. SLA research that has come out of this turn has led to numerous advances. To take but a few examples, underlying concepts and constructs have been (re)defined (e.g., \citealt{Pallotti2009} on the construct of complexity-accuracy-fluency), certain well-established ways of doing things have been questioned (e.g., \citegen{PlonskyOswald2017} plea to move away from ANOVA), and new approaches have been developed and championed (see the numerous recent special issues devoted to both wide and narrow methodological issues: \citealt{NorrisEtAl2015,ChoiRichards2016,DeCostaEtAl2017, EdmondsEtAlinpress}). As a result, the methodological landscape in SLA is arguably more diverse than ever before, with \citet[5]{Ortega2013} identifying the increase in “research methodological prowess” as one of the noticeable trends in SLA research.

According to  \citet[214]{KingMackey2016}, the field of SLA

is in its prime. It has left behind the largely unproductive, so-called 'paradigm wars' between those supporting quantitative and qualitative approaches. Both cognitively and socially oriented researchers are showing greater awareness of the importance of incorporating a range of perspectives. The field is pushing methodological boundaries in many directions.

The pushing referred to by King and Mackey is taking many forms, including cross-disciplinary pollination and collaboration (e.g., \citealt{DuffByrnes2019}), the use of mixed-methods, leading to an attitude of {“methodological inclusivity” \citep[478]{Römer2019}, and a growing number of scientific publications that tackle methodological issues head on. In this final category, researchers generally aim to stimulate discussion and potentially initiate change, be this through discussion papers, such as \citet{TheDouglasFirGroup2016} or \citet{Young2018}, or with empirical studies (often through reanalyses of previous published data or meta-analyses), which serve to concretely demonstrate the import and impact of methodological choices (\citealt{SantosEtAl2008,LeeserSunderman2016,EdmondsGudmestad2018,Solon2018}).}


{With the current collected volume, we aim to contribute to this focus on methodological issues. Specifically, we bring together a collection of seven chapters, each of which} provides a new angle on the treatment or interpretation of lang\-uage-learning data, a crucial issue in the building of knowledge in the field of SLA. {Three main lines of reflection are pursued in these chapters.}



{The first concerns the question of how comparisons to a baseline norm can be carried out in L2 research, as well as what norm might be best adopted. In the present volume, this question is addressed from two novel standpoints: the question of how to identify interlanguage forms in the dialect-rich environment of Norway, which provides many different input forms for the same concept (}{{Evenstad Emilsen \& Søfteland}}{) and the questioning of a general native baseline in event-related potential (ERP) studies (Pélissier).}



{The second line of reflection, broadly speaking, concerns epistemological stance in research design. By epistemological stance, we refer to a researcher’s view about what constitutes knowledge in a given field. One common epistemological tension in the field of SLA opposes two visions of language learning: “}Is learning like acquiring stuff or is it like doing things?” \citep[45]{Young2018}. These two visions lead to different positions on how to study language learning and even as regards to what is ultimately worthy of study. {Issues connected to the role of epistemological stance are visible in two chapters. Whereas Watorek, Rast, Yu, Trévisiol, Majdoub, Guan \& Huang reflect on how to carry out a conceptual replication, thereby holding constant their epistemological understanding of the phenomenon under study (namely, L2 acquisition from the very initial stages), Gudmestad purposefully sets out to follow to its logical conclusion a shift in epistemological stance.}



{The third question grapples with more technical issues surrounding the annotation, coding, and interpretation of data,} especially when faced with ambiguous interlanguage forms.{ Issues identified involve both multimodal data (Hilton \& Osborne; Scheuer \& Horgues) and difficulties specific to the transcription of oral data (Leclercq).}


{In the first chapter,} {{Evenstad Emilsen \& Søfteland offer reflections on SLA in a dialect-rich environment. Such environments have received little explicit attention in the research literature, and yet they entail challenges for both learners and for researchers. For learners, the co-existence of multiple dialects provides an arguably more complex input, one in which numerous forms co-exist to express the same function. For researchers, making data coding and analysis decisions about learner production is particularly challenging, as forms found in interlanguage use may not correspond to the dominant dialect, but may be present in other varieties. The authors detail the challenges facing researchers, providing several examples. They highlight the difficulties inherent in determining whether forms produced by learners are evidence of sociolinguistic variation (i.e., variation present in the input) or instances of interlanguage variation.} }

{{Pélissier’s contribution questions the comparison of native and non-native performance in online processing studies involving ERPs. The author shows that although a large body of research into native language processing has identified a biphasic ERP pattern when native speakers are asked to process syntactic violations, recent research has called this pattern into question, showing instead that there is substantial inter-individual variability among native speakers. More specifically, when it comes to syntactic violations, most individuals show only one of the two components of the biphasic pattern. For the field of SLA, traditionally preoccupied with comparing native and non-native performance, this finding begs the question of how we might meaningfully compare learners and native speakers. Pélissier explores two approaches that hold some promise insofar as they allow researchers to account for individual variability: the Response Magnitude Index and the Response Dominance Index \citep{TannerEtAl2014}. The target structure in Pélissier’s study is past-tense morphology with auxiliaries in English. Results show that the Response Dominance Index, but not the Response Magnitude Index, is useful in accounting for the data analysed.}}

{{In the third chapter, Watorek and colleagues provide a detailed presentation of the ambitious VILLA project (}}\textit{Varieties of Initial Learners in Language Acquisition: Controlled classroom input and elementary forms of linguistic organisation}{{). This project seeks to provide insight into language acquisition in the first hours of exposure to a new language. In the original VILLA project, Polish is the target language, with learners having either Dutch, English, French, German or Italian as their native language. The contribution included in this volume reflects on three conceptual replications of the VILLA project, in order to study the initial acquisition of Modern Standard Arabic, Mandarin Chinese, and Japanese by native French speakers. The goal of the conceptual replications is to contribute additional insight into language learning starting from first exposure, but with typologically diverse languages. This diversity requires the authors to reconsider the target of learning (nominal morphology, in the original project), the variables controlled for (transparency and frequency), as well as the way of assessing learning. The reflection offered by the authors raises the intriguing question of comparability when transposing research design and questions to study new language combinations.}}

{Gudmestad’s chapter directly addresses the oft-ignored issue of epistemological stance. In other words, she engages with ``what counts'' as knowledge in SLA. Using the concrete example of grammatical gender in L2 Spanish, she highlights the fact that there exist (at least) two different epistemological understandings as to what production of gender-marked modifiers reveals about interlanguage. One position (exemplified in Gudmestad’s previous work) considers all instances of gender marking to reveal the same underlying process, regardless of whether the modifier in question is an adjective or a determiner. The second position sees two different processes at work: on the one hand, the gender marked on determiners is thought to reflect the gender attributed by the speaker to the noun in question (a lexical property) and, on the other, gender marked on adjectives reveals the speaker’s ability to compute morphosyntactic agreement. In her chapter, Gudmestad departs from her original stance in order to ``try on'' the second position in a reanalysis of data originally published in \citet{GudmestadEtAl2019}. She thereby explores what is gained by adopting new ways of seeing data. In so doing, Gudmestad essentially participates in what \citet[214]{KingMackey2016} term “layering”:} “Layering involves considering theory as well as practice and, in particular, considering varied epistemological stances every time one looks at a traditional problem.”

The next chapter provides a concrete and critical reflection on how the tool EXMARaLDA can be profitably used to carry out multi-tiered annotation of classroom data. Hilton and Osborne report on part of an exploratory study that took place in English classes held in two French elementary schools. After detailing the development of their multi-layered approach to transcribing and annotating three weeks of language lessons, the authors focus on data from one lesson from each classroom in order to demonstrate how conducting analyses at different levels of annotation may lead to the identification of the differences in the two learning environments that triggered different learning outcomes for the students (regarding memorization of new vocabulary and utterance construction) Although the authors highlight that the analyses are limited in scale and thus cannot be used to suggest pedagogical implications, they demonstrate that the two classrooms are not equally effective, which is visible, for example, in the organisation of pupil and teacher talk.

Chapter six focuses on how theory, data coding, and data transcription intersect. To accomplish this goal, Leclercq uses the example of verb-final [e] in L2 French. Verb-final [e] in French can correspond to the infinitive form (\textit{parler} ‘to speak’), imperfective forms (e.g., \textit{parlais} ‘(you) speak’, \textit{parlait} ‘(s/he) speaks’), the first-person simple past form (\textit{parlai} ‘(I) spoke’), and various forms of the past participle (\textit{parlé, parlés, parlée, parlées}). In other words, one spoken form – [p{aʁle}] – is highly homophonous. This leads to a clear challenge for any researcher working on oral productions in L2 French. How does one transcribe a form like [p{aʁle}] when produced by a learner? Leclercq takes up this thorny issue and critically details how other studies in SLA research have dealt with it. She concludes by showing that some transcription choices result from a premature categorisation of the data, often reflecting theoretical positioning and potentially introducing interpretative bias.

The volume closes with a chapter devoted to identifying and reflecting on potential pitfalls involved in analysing data from English-French tandem conversations. Scheuer and Horgues report on data collected from 21 tandem pairs during a semester-long programme at a French university. Each tandem is made up of a native speaker of French and of English and was recorded on two occasions (once at the beginning and once at the end of the semester). For each recording, approximately half of the speaking time is in each of the two languages. The authors use these data to explore corrective feedback and communication breakdowns, addressing, among other things, which member of the tandem initiated the feedback or signalled the breakdown and what type of issue (lexis, pronunciation, syntax, etc.) led to the feedback or breakdown. The authors offer a thought-provoking discussion on the difficulties involved in determining both what constitutes corrective feedback or comprehension breakdowns and in pinpointing what linguistic issue was the cause (or causes) for either. They thus provide clear and concrete examples of dealing with ambiguity in learner data in an L2 analysis.

The seven chapters brought together in this volume offer original and timely contributions on the role of (native-speaker) norms in L2 analyses, on the impact of epistemological stance, and on the challenges of transcription and annotation of language-learning data. In addressing these issues, the researchers rely on a variety of methodological practices and highlight in their chapters the import of methodological choice. These choices have a far-reaching impact, as they constrain and orient what observations can be made in research and what conclusions are ultimately drawn. We hope to have demonstrated with this volume that reflecting on these decisions – making them explicit and holding them up to study – is indeed a valuable enterprise. 

\sloppy\printbibliography[heading=subbibliography,notkeyword=this]
\end{document}
